\chapter{Introduzione}

\section{Che cos'è la firma digitale?}

\subsection{Dal punto di vista generale}

La firma digitale rappresenta una tecnica, fondata su precisi principi matematici, che ha lo scopo di dimostrare l'autenticità di un documento digitale, garantendo:
\begin{itemize}
	\item l'integrità dei dati in esso contenuti;
	\item l'autenticità delle informazioni relative al sottoscrittore;
	\item la non alterabilità del documento;
	\item le non ripudiabilità della firma.
\end{itemize}
Le caratteristiche sovraelencate implicano che, il sottoscrittore una volta apposta la firma:
\begin{itemize}
	\item non potrà disconoscere il documento, il quale non potrà essere assolutamente modificato (una eventuale modifica ne annulla la validità);
	\item diventa l'unico titolare del certificato, dal momento che è in possesso delle credenziali per accedervi.
\end{itemize}

\subsection{Dal punto di vista tecnico}

La firma può essere realizzata attraverso protocolli crittografici simmetrici (a chiave privata) o asimmetrici (a chiave pubblica). La maggior parte delle applicazioni utilizzano procolli asimmetrici, perché permettono di creare soluzioni più semplici e computazionalmente efficienti rispetto agli altri.

\newpage

\section{Firma digitale vs analogica}

Facendo una attenta ricerca, è possibile stabilire che, le proprietà godute dalle firme digitali sono presenti anche in quelle analogiche, eppure è risaputo che le prime sono decisamente più sicure ed affidabili rispetto alle altre, questo perché:
\begin{itemize}
	\item la firma è falsificabile solo attraverso la conoscenza della chiave privata del firmatario, inoltre, modificando anche solo un bit del documento, esso perde di validità (per il motivo analogo, non è nemmeno riutilizzabile), di conseguenza è una tutela molto più forte rispetto alla firma analogica;
	\item l'autore non può negare la paternità della dichiarazione presente nel documento siccome, al momento della firma era l'unico in possesso della chiave privata.
	\item la firma dipende fortemente dal documento sul quale viene posta.
\end{itemize}
